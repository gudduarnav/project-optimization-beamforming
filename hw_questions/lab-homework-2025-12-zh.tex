\documentclass[11pt,letterpaper]{ctexart}
\usepackage[margin=0.65in]{geometry}
\usepackage[T1]{fontenc}
\usepackage[utf8]{inputenc}
\usepackage{amsmath,amssymb}
\usepackage{bm}
\usepackage{xcolor}
\usepackage[most]{tcolorbox}
\usepackage{fancyhdr}
\usepackage{microtype}
\usepackage{hyperref}

\definecolor{zhprimary}{RGB}{145,0,110}
\definecolor{zhsecondary}{RGB}{0,140,160}
\definecolor{zhaccent}{RGB}{255,140,0}
\definecolor{zhlight}{RGB}{252,246,255}

\setCJKmainfont{Noto Serif CJK TC}

\hypersetup{
  colorlinks=true,
  linkcolor=zhprimary,
  urlcolor=zhsecondary,
  citecolor=zhprimary
}

\tcbset{
  zhproblemstyle/.style={
    enhanced,
    breakable,
    colframe=zhsecondary,
    colback=zhlight,
    boxrule=0.9pt,
    arc=4pt,
    outer arc=4pt,
    fonttitle=\bfseries,
    coltitle=white,
    colbacktitle=zhprimary,
    attach boxed title to top left={yshift=-2mm,xshift=2mm},
    boxed title style={
      boxrule=0pt,
      sharp corners,
      left=4mm,right=4mm,top=1mm,bottom=1mm
    }
  }
}

\newtcolorbox{zhProblemBox}[1][]{zhproblemstyle,#1}

\newtcolorbox{zhtitleBox}{
  enhanced,
  colback=white,
  colframe=zhprimary,
  boxrule=1pt,
  arc=5pt,
  left=8mm,right=8mm,top=4mm,bottom=5mm,
  borderline west={2mm}{0pt}{zhaccent},
  overlay={
    \path[fill=zhsecondary!20!white]
      (frame.north west) -- ([xshift=1.2cm]frame.north west) --
      ([yshift=-0.25cm]frame.north east) -- (frame.north east) -- cycle;
  }
}

\pagestyle{fancy}
\fancyhf{}
\fancyhead[L]{\small\color{zhprimary} 最適化理論實驗作業}
\fancyhead[R]{\small\color{zhsecondary} CVX 專題(中文說明)}
\cfoot{\thepage}
\renewcommand{\headrulewidth}{0.4pt}
\renewcommand{\headrule}{\hbox to\headwidth{\color{zhsecondary}\leaders\hrule height \headrulewidth\hfill}}
\setlength{\headheight}{15pt}

\setlength{\parindent}{0pt}
\setlength{\parskip}{4pt}

\begin{document}

\begin{zhtitleBox}
  \centering
  {\small\textsc{Optimization Theory Course}}\\[0.2em]
  {\LARGE\bfseries 最適化理論實驗作業 -- CVX 專題}\\[0.35em]
  {\large\color{zhprimary} 實驗作業題目(繁體中文說明)}\\[0.35em]
  {\normalsize 繳交截止日期:2025 年 12 月 2 日}\\[0.6em]
  {\footnotesize\color{zhprimary}\itshape
    本文件為英文作業說明的繁體中文翻譯版本,僅供閱讀與輔助理解之用。\\
    正式繳交的作業解答與報告必須以英文撰寫。}\\[0.4em]
  {\footnotesize\color{zhsecondary}\itshape
    原始英文作業與 MATLAB(CVX)/Python(CVXPY)程式碼請參考 GitHub 倉庫:\\
    \url{https://github.com/gudduarnav/project-optimization-beamforming}}
\end{zhtitleBox}

\vspace{1.2em}

\begin{zhProblemBox}[title={問題 1 -- 帶界最小平方法(Bound-Constrained Least-Squares)}]
考慮以下帶界(上下界)限制的最小平方法問題:
\begin{align}
  \min_{\bm{x} \in \mathbb{R}^n} \;\; \lVert A \bm{x} - b \rVert_2^2
  \quad \text{s.t.} \quad 0.1 \le \bm{x} \le 3.14,
\end{align}
其中 $A \in \mathbb{R}^{m \times n}$,$b \in \mathbb{R}^m$,向量 $\ell, u \in \mathbb{R}^n$ 分別為
變數向量 $\bm{x}$ 的一般下界與上界;在本次作業中,假設 $\bm{x}$ 的每一個分量皆滿足
$0.1 \le x_i \le 3.14$。

\medskip
\noindent
\textbf{作業要求:} 使用 CVX(或其他類似的凸優化工具箱)將上述問題寫成標準形式並求解,
求出最優解向量 $\bm{x}^\star$,並簡要說明在 $0.1 \le x_i \le 3.14$ 的限制下,你對解
$\bm{x}^\star$ 各分量可以得到哪些推論(例如哪些分量貼近界線,以及這對殘差
$\lVert A \bm{x} - b \rVert_2^2$ 有何影響)。
\end{zhProblemBox}

\vspace{1em}

\begin{zhProblemBox}[title={問題 2 -- 平均旁瓣能量最小化(Average Sidelobe Energy Minimization)}]
考慮一個具 $M$ 個天線元件的均勻線性陣列(ULA)波束成形器,其複數權重向量為
$\bm{w} \in \mathbb{C}^M$。令 $\bm{P} \in \mathbb{C}^{M \times M}$ 為一個半正定矩陣,用來描述在某一
角度區間內的平均旁瓣能量;令 $\bm{a}(\theta_0)$ 為指向期望主波束方向 $\theta_0$ 的陣列響應向量
(steering vector),在本次作業中令期望主波束角度為 $\theta_0 = +10^\circ$。我們希望尋找可以
最小化平均旁瓣能量的波束成形權重:
\begin{align}
  \min_{\bm{w} \in \mathbb{C}^M} \;\; \bm{w}^H \bm{P} \bm{w}
  \quad \text{s.t.} \quad \bm{w}^H \bm{a}(\theta_0) = 1.
\end{align}

\noindent
\textbf{提示:} 在 CVX 中,可使用 \verb|quad_form(w, P)| 來表示二次型 $\bm{w}^H \bm{P} \bm{w}$。

\medskip
\noindent
(a) 在 CVX 中實作上述優化問題,並針對 $\theta_0 = +10^\circ$ 求得最優的波束成形向量 $\bm{w}$。\\
(b) 繪出角度頻譜(例如波束圖 / beampattern),說明旁瓣行為與特性,特別是主瓣在
    $\theta_0 = +10^\circ$ 附近的形狀與旁瓣抑制效果,並寫出你的觀察與推論。
\end{zhProblemBox}

\vspace{1em}

\begin{zhProblemBox}[title={問題 3 -- 最壞情況旁瓣最小化(Worst-Case Sidelobe Minimization)}]
接著考慮魯棒波束成形設計,我們希望在一組旁瓣角度集合
$\{\theta_1, \ldots, \theta_L\}$ 上最小化\emph{最壞情況}的旁瓣電平。令 $\bm{a}(\theta)$ 表示角度
$\theta$ 處的陣列響應向量。引入輔助實數變數 $t$,考慮以下優化問題:
\begin{align}
  \min_{\bm{w}, t} \;\; t
  \quad \text{s.t.} \quad
  \lvert \bm{w}^H \bm{a}(\theta_\ell) \rvert^2 \le t,\quad \ell = 1, \ldots, L, \qquad
  \bm{w}^H \bm{a}(\theta_0) = 1.
\end{align}

\noindent
(a) 給定天線數 $M$ 與旁瓣角度區間,求解上述問題並繪出對應的角度頻譜。
將結果與問題 2 的設計進行比較。\\
(b) 將天線數設為 $M = 32$,重新設計波束成形器並繪出角度頻譜。說明增加 $M$(相較於 (a))對主瓣寬度
與旁瓣電平造成的影響。\\
(c) 考慮以下兩種不同的期望主瓣區間:
\begin{itemize}
  \item[(i)] $\theta_\ell = -10^\circ$, $\theta_u = +10^\circ$,
  \item[(ii)] $\theta_\ell = 45^\circ$, $\theta_u = 55^\circ$。
\end{itemize}
對每一種主瓣區間,設計對應的波束成形器並繪出角度頻譜,說明主瓣寬度與旁瓣行為之間的取捨關係。
\end{zhProblemBox}

\vspace{1em}

\begin{zhProblemBox}[title={問題 4 -- 含 SINR 約束的下行波束成形(Downlink Beamforming with SINR Constraints)}]
考慮一個基地台(BS)具備 $M$ 根天線,同時為 $N$ 個行動用戶(MS)進行下行傳輸。每個用戶僅有單一天線。
令 $\bm{h}_k \in \mathbb{C}^M$ 為從基地台到第 $k$ 位用戶的通道向量,$\bm{w}_k \in \mathbb{C}^M$ 為
第 $k$ 位用戶的波束成形向量($k = 1, \ldots, N$)。假設用戶 $k$ 的接收訊號同時受到多使用者干擾與
雜訊(變異數為 $\sigma_k^2$)影響,且用戶 $k$ 的 SINR 要求為 $\gamma_k > 0$。用戶 $k$ 的下行 SINR
可寫為
\begin{align}
  \mathrm{SINR}_k
  = \frac{\bigl\lvert \bm{h}_k^H \bm{w}_k \bigr\rvert^2}
         {\displaystyle \sum_{\substack{j = 1 \\ j \ne k}}^{N}
            \bigl\lvert \bm{h}_k^H \bm{w}_j \bigr\rvert^2 + \sigma_k^2 }.
\end{align}

\noindent
我們希望在滿足所有用戶 SINR 要求的情況下,最小化基地台的總發射功率:
\begin{align}
  \min_{\{\bm{w}_k\}_{k=1}^N} \;\; \sum_{k=1}^N \lVert \bm{w}_k \rVert_2^2
  \quad \text{s.t.} \quad
  \frac{\bigl\lvert \bm{h}_k^H \bm{w}_k \bigr\rvert^2}
       {\displaystyle \sum_{\substack{j = 1 \\ j \ne k}}^{N}
          \bigl\lvert \bm{h}_k^H \bm{w}_j \bigr\rvert^2 + \sigma_k^2 }
  \ge \gamma_k,\quad k = 1, \ldots, N.
\end{align}

\medskip
\noindent
(a) 將上述問題重新寫成適合在 CVX 中求解的凸優化問題(例如以二階錐規劃 SOCP 形式表示)。\\
(b) 以 $M = 32$ 根天線與 $N = 5$ 個行動用戶進行數值模擬,假設每位用戶的雜訊功率對應
    至 $-70\,\mathrm{dBm}$ 的雜訊變異數,且所有用戶的 SINR 目標皆為 $\gamma_k = 10\,\mathrm{dB}$。
    驗證在最小總發射功率的設計下,所有用戶的 SINR 均滿足其目標值,並簡要說明基地台對各用戶
    的功率分配情形與你的觀察。
\end{zhProblemBox}

\end{document}
